\documentclass[letter]{cv-template}
\usepackage{enumitem}
\usepackage{hyperref}
\usepackage{tikz}
\setlist{itemsep=1pt}
\hypersetup{
    colorlinks=true,
    urlcolor=black
}


\name{Khoa Ho Anh}
\address{}{719 Ta Quang Buu Street, W.4, D.8, HCM City}
\phone{(+84) 368-367-501}
\email{hoanhkhoakhtn@gmail.com}
\github{https://github.com/AKKHTN2131999}


\begin{document}

    \displayheader \vspace{0.25 cm}

   

    \createsection{Summary}\\
        % \vspace*{7pt}
        \textit{I am a former student of the University of Natural Science (under the Vietnam National University, Ho Chi Minh City). My major is computer science. With my love for Artificial Intelligence, I have been constantly studying and working hard in this field. I have 1 year experience working with Chatbot. I hope that a suitable job in the not-too-distance future will continue to fuel my passion.}\\\\

    \createsection{Education}

        \createtitle{Bachelor of Science, Computer Science}{Sep 2017 -- May 2022}
        {VNUHCM - University of Science | GPA: 3.24}{HCMC, Vietnam}{
            \begin{itemize}
                \item \textit{Thesis}: Integrating WordNet into neural network machine translation  (July 2021). 
            \end{itemize} \leavevmode
        }




       
    \createsection{AI Engineer at Zalo}
    
        \createproject{Chatbot with Rasa framework}{Dec 2021 -- July 2022}
        {}{
            \begin{itemize}
                \setlength\itemsep{0.25pt}
                \item Build training data, scripts,  policies and rules in rasa for a specific chatbot task, e.g. collect customer information, confirm OPT, Q\&A, ...
                \item Data augmentation in rasa flow.
                \item Improve response processing speed when the conversation is long.
                \item Use the Rasa X tool to deploy a chatbot for training and data collection.
                \item Using the API command of Rasa X for the process that using Python just got easier. 
                \item Gaining knowledge: How to create a chatbot with Rasa and personalize on demand.
            \end{itemize} \leavevmode
        }
    \createproject{General OCR for Vietnamese}{July 2022 -- now}
        {}{
            \begin{itemize}
                \setlength\itemsep{0.25pt}
                \item Create synthesis data.
                \item Research and experiment with deep model
            \end{itemize} \leavevmode
        }

    \createsection{Thesis}

        \createproject{Integrating WordNet into neural network machine translation}{Dec 2020 -- July 2021}
        {}{
            \begin{itemize}
                \setlength\itemsep{0.25pt}
                \item Score: 10.0
                \item Improved LSW algorithm and increased 0.92 BLEU score.
                \item Optimize wnet2vec memory and training time and achieve better word embbedding efficiency.
                \item Cluster synsets (set of synonyms in Wordnet) and feed this information into word representation. Helps improve 1.55 BLEU.
                \item Framework: Torch
                \item Gaining knowledge: Know how to research and experiment in deep learning.
            \end{itemize} \leavevmode
        }
    \createsection{Skills}
        \textit{Programing languages}: Python, Bash Shell \\
        \textit{Libraries \& Frameworks}: PyTorch, TensorFlow/Keras, Scikit-learn, Huggingface, Rasa \\
        \textit{Languagues}: Vietnamese, English (reading skills may be required)
\end{document}